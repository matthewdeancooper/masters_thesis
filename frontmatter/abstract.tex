\chapter{Abstract} \label{ch:abstract}

\textbf{Purpose:} Accurate contouring is a critical aspect of safe and effective
treatment delivery in radiotherapy (RT). Current limitations in clinical practice
include large intra and inter-observer variances (IOV), as well time delays in
both contour generation and correction. This study designed and evaluated a 2D U-Net architecture with
two primary aims: 
1) Develop a pelvic imaging quality assurance tool for use in
RT, comparing model predictions with expert contours. 2) Automate vacuum bag
segmentation for canine RT.

\textbf{Method:} This study presents two independently trained models, and
assesses the performance of common semantic segmentation loss functions in each
case. The original U-Net architecture developed by Ronneberger et al. was expanded
integrating recent network modifications that indicated improved performance in
the literature. In addition to reporting dice similarity coefficients (DSC), organ-specific tolerances (representative of expert IOV) are utilised for bladder
and rectum contouring in Nikolov et al's surface dice similarity coefficient
(sDSC).

\textbf{Results:} Three contours were produced from pelvic imaging CT scans:
Patient contours were measured with DSC $>0.99(0.001)$, bladder contours with
DSC 0.86(0.2) and sDSC 0.87(0.1), and rectum contours with DSC 0.67(0.1)
and sDSC 0.92(0.14).  Additionally, vacuum bag contours from canine imaging CT
scans were measured with DSC 0.95(0.001). Weighted DSC was the only loss function
that optimised for all organs considered in pelvic imaging - due to a
significant pixel-wise class imbalance between structures.

\textbf{Conclusion:} Patient contours did not deviate outside tolerances and are
viable for use within the QA tool in their current form.  Bladder and rectum
segmentation may improve with a broader dataset (19 patient scans were used).
The vacuum bag model should proceed to acceptance testing for clinical
implementation.
