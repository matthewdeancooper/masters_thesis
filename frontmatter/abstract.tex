\chapter{Abstract} \label{ch:abstract}

\textbf{Introduction:}
% Accurate contouring is a critical aspect of safe and effective
% treatment delivery in radiotherapy (RT). Current limitations in clinical
% practice include large intra and inter-observer variances (IOV), as well as time
% delays in both contour generation and correction. This study designed and
% evaluated a 2D U-Net architecture with two primary aims: 1) Develop a pelvic
% imaging quality assurance tool for use in RT, comparing model predictions with
% expert contours. 2) Automate vacuum bag segmentation for canine RT.
Accurate contouring is a critical aspect of safe and effective treatment
delivery in radiotherapy (RT). Current limitations in clinical practice include
large intra and inter-observer variances (IOV), as well as time delays in both
contour generation and correction. Additionally, the AAPM TG275 risk assessment
recommendations highlight multiple human-factor failure modes in RT. An
automated contour quality assurance (QA) review has the potential to manage some
prominent hazards identified by the task group. This study designed and
evaluated a 2D U-Net architecture with two primary aims: 1) Develop a pelvic
imaging quality QA tool for use in RT, comparing model predictions with expert
contours to identify macro contouring errors. 2) Automate vacuum bag
segmentation for canine RT.

\textbf{Method:}
% This study presents two independently trained models and
% assesses the performance of common semantic segmentation loss functions in each
% case. The original U-Net architecture developed by Ronneberger et al. was
% expanded by integrating recent network modifications that indicated improved
% performance in the literature. In addition to reporting dice similarity
% coefficients (DSC), organ-specific tolerances (representative of expert IOV) are
% utilised for bladder and rectum contouring to quantify Nikolov et al.'s surface
% dice similarity coefficient (sDSC) - reported to have a stronger
% correlation with contour correction times when compared to traditional metrics.
% The pelvic imaging dataset consisted of 15 patients undergoing RT for prostate
% cancer, and was split into training, validation, and testing subsets of 12, 2,
% and 1, respectively. The canine imaging dataset included 26 patients selected
% from varied RT treatment groups, split into subsets of 21, 3, and 2, respectively.
This study presents two independently trained models and assesses the
performance of common semantic segmentation loss functions in each case. The
original U-Net architecture developed by Ronneberger et al. was expanded by
integrating recent network modifications that indicated improved performance in
the literature. In addition to reporting dice similarity coefficients (DSC),
organ-specific tolerances (representative of expert IOV) are utilised for
bladder and rectum contouring to quantify Nikolov et al.'s surface dice
similarity coefficient (sDSC) - reported to have a stronger correlation with
contour correction times when compared to traditional metrics. The pelvic
imaging dataset consisted of 15 patients undergoing RT for prostate cancer, and
was split into training, validation, and testing subsets of 12, 2, and 1,
respectively. The canine imaging dataset included 26 patients selected from
varied RT treatment groups, split into subsets of 21, 3, and 2, respectively.

\textbf{Results:}\footnote{Notation: $\bar{x}(\sigma)$ corresponds to a
measurement with mean value $\bar{x}$ and standard deviation $\sigma$.} Three
contours were produced from pelvic CT: Patient contours were measured with DSC
$0.998(0.001)$, bladder contours with DSC $0.9(0.2)$ and sDSC $0.9(0.2)$, and
rectum contours with DSC $0.7(0.1)$ and sDSC $0.9(0.1)$. Vacuum bag contours
from canine CT were measured with DSC $0.952(0.001)$. Weighted DSC was the only
loss function that optimised for all organs considered in pelvic imaging -
due to a significant pixel-wise class imbalance between structures.

\textbf{Conclusion:}
% Patient contours did not deviate outside of tolerances and
% are viable for use within the QA tool in their current form. Bladder and rectum
% segmentation may improve with a broader dataset. Additionally, a 3D model may
% provide axial context to improve segmentation on rectal regions containing gas.
% In its current implementation, the surface dice similarity coefficient accepts
% only binary input data; hence, gradients are not defined when used as a cost
% function. Developing a soft surrogate may allow for the direct optimisation of
% this metric. The canine imaging model was successfully deployed to clinic under
% a prototype warning, where preliminary acceptance testing has shown a
% performance improvement of approximately 30 minutes per patient. It is currently
% being utilised on all new canine patients undergoing RT.
Patient contours did not deviate outside of tolerances and are viable for use
within the QA tool in their current form. Bladder and rectum segmentation may
improve with a broader dataset. Additionally, a 3D model may provide axial
context to improve segmentation on rectal regions containing gas. In its current
implementation, the surface dice similarity coefficient accepts only binary
input data; hence, gradients are not defined when used as a cost function.
Developing a soft surrogate may allow for the direct optimisation of this
metric. The canine imaging model was successfully deployed to clinic under a
prototype warning that requested manual verification before use. Preliminary
acceptance testing has shown a performance improvement of approximately 30
minutes per patient. It is currently being utilised on all new canine patients
undergoing RT.

A video overview of the clinical implementation is available:\\
\href{https://docs.pymedphys.com/background/autocontouring.html}{docs.pymedphys/com/background/autocontouring}