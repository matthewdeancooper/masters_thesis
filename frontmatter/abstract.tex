\chapter{Abstract}
\label{ch:abstract}

\textbf{Purpose:} Accurate contouring is a critical aspect of safe and effective treatment delivery in radiotherapy (RT). Current limitation in clinical practice include: Large inter and intra-observer variances, as well a time delays in both contour generation and correction - that act as barriers to the implementation of adaptive RT. This study designs and evaluates a 2D U-net architecture with two primary aims: 1) Develop a pelvic imaging quality assurance tool for use in RT, comparing model predictions with expert contours. 2) Automate vacuum bag segmentation for canine RT.

\textbf{Method:}
 We present two independently trained models in this study, and assess the performance of common semantic segmentation loss functions in each case. We expand on the original architecture developed by Ronneberger et al. by integrating recent network modification that have shown improved performance in the literature. In addition to reporting dice similarity (DSC), we utilise organ-specific tolerances representative of expert IOV for the bladder and rectum as parameters in Nikolov et al's surface dice similarity coefficient (sDSC).

\textbf{Results:}
3 contours were produced from pelvic imaging CT scans: Patient contours were measured with mean DSC 0.99(0.01), bladder contours with mean DSC 0.86(0.22) and mean sDSC 0.87(0.18), and rectum contours with mean DSC 0.67(0.12) and mean sDSC 0.92(0.14). Additionally, vacuum bag contours from canine imaging CT scans were measured with mean DSC 0.95(0.01). Weighted DSC was the only loss function that optimised for all organs considered in pelvic imaging - due to a significant class imbalance.

\textbf{Conclusion:}
Patient contours were excellent. We suspect a broader dataset may improve bladder and rectum segmentations (19 patient scans were used). The vacuum bag model should proceed to acceptance testing for clinical implementation.