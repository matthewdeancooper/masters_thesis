\chapter{Conclusion}
\label{ch:conclusion}

In this study, we attempted 2D U-net architecture with a small dataset and image augmentation, over a variety of standard loss functions used in semantic segmentation tasks. Two models were developed: Model 1 aimed to contour patient, bladder, and rectum structure in pelvic CT images, to provide a QA tool for background monitoring of IOV in RT. Additionally, we provided surface dice similarity coefficients for the bladder and rectum contours, with organ-specific tolerances at the 95th percentile mean surface distance between expert observers. Model 2 aimed to automate the contouring of vacuum bags in canine imaging for RT, a time-consuming structure that is delineated manually as SASH veterinary clinic.

Weighted soft DSC loss was selected for the pelvic imaging model as it was the only loss that overcame the class imbalance in our data to optimise for all OARs in the model output. Patient contours were excellent, with a DSC of 0.998. We suspect more data will be required to improve bladder and rectum segmentation for use as a QA tool, with DSCs of 0.860 and 0.670, respectively. State-of-the-art implementations employ dataset on the order of 1000 patients, compared to the 16 included in our study. However, the surface dice similarity coefficient indicated that the majority of bladder and rectum surfaces were within expert IOV, with sDSCs of 0.876 and 0.922, respectively - indicating the potential for time saving in contour correction.

Soft DSC loss was selected for the canine imaging model, which produced (likely) acceptable vacuum bag contouring with a DSC of 0.952. As this model has the potential to save 30 minutes of planning time per patient, further work will involve clinical acceptance testing and implementation.


