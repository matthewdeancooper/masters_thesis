\chapter{Conclusion}
\label{ch:conclusion}

This study attempted 2D U-Net architecture with a small dataset and image
augmentation. Performance was assessed under a variety of standard loss
functions used in semantic segmentation tasks. Two models were developed: Model
1 aimed to contour the patient, bladder, and rectum structures in pelvic CT
images, to provide a QA tool for background contour monitoring RT. Additionally,
sDSC was calculated for the bladder and rectum with organ-specific tolerances at
the 95th percentile mean surface distance between expert observers. Model 2
aimed to automate the contouring of vacuum bags in canine imaging for RT, a
time-consuming structure that is delineated manually at SASH veterinary clinic.

Weighted soft DSC loss was selected for the pelvic imaging model as it was the
only loss that overcame the class imbalance in our data to optimise for all
OARs. Patient contours were clinically acceptable with DSC of $>0.998(0.001)$.
However, it is suspected that more data will be required to improve bladder and
rectum segmentation, before it is feasible to use this model as a clinical QA
tool. Bladder and rectum contours were measured with DSCs of $0.9(0.2)$ and
$0.7(0.1)$, respectively.

However, the surface dice similarity coefficient indicated that the majority of
bladder and rectum surfaces were within expert IOV, with sDSCs of $0.9(0.2)$ and
$0.9(0.1)$, respectively. However, 18\% of of rectum contours were missed, with each
example containing a hollow region. Further training on similar cases is likely
to improve detection rates.

Soft DSC loss was selected for the canine imaging model, which produced
clinically acceptable vacuum bag contouring with a DSC of $0.952(0.001)$. Results
indicated that the maximum measured MSD 0.82 mm corresponded to an equivalent
in-water depth of 0.08 mm; and hence, a negligible maximum shift in dose
distribution ($<0.03\%$) at 10cm depth. As this model has the potential to save
30 minutes of planning time per patient, further work will involve clinical
acceptance testing and implementation - under the condition that a human
operator will still investigate all contouring prior to treatment. It is
expected that little to no correction will be required during routine clinical
use.

\textbf{Update:} The canine imaging model was successfully deployed to clinic
under a prototype warning, where preliminary acceptance testing has indeed shown
a performance improvement of approximately 30 minutes per patient. It is
currently being utilised on all new canine patients undergoing RT.