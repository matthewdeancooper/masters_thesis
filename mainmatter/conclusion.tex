\chapter{Conclusion}
\label{ch:conclusion}

This study attempted 2D U-Net architecture with a small dataset and image augmentation. Performance was assessed under a variety of standard loss functions used in semantic segmentation tasks. Two models were developed: Model 1 aimed to contour the patient, bladder, and rectum structures in pelvic CT images, to provide a QA tool for background contour monitoring RT. Additionally, sDSC was calculated for the bladder and rectum with organ-specific tolerances at the 95th percentile mean surface distance between expert observers. Model 2 aimed to automate the contouring of vacuum bags in canine imaging for RT, a time-consuming structure that is delineated manually at SASH veterinary clinic.

Weighted soft DSC loss was selected for the pelvic imaging model as it was the only loss that overcame the class imbalance in our data to optimise for all OARs. Patient contours were clinically acceptable with DSC of $>0.99(0.001)$. However, it is suspected that more data will be required to improve bladder and rectum segmentation, before it is feasible to use this model as a clinical QA tool. Bladder and rectum contours were measured with DSCs of 0.86(0.2) and 0.67(0.1), respectively.

However, the surface dice similarity coefficient indicated that the majority of bladder and rectum surfaces were within expert IOV, with sDSCs of 0.876 and 0.922, respectively. However, 18\% of of rectum contours were missed, with each example containing a hollow region. Further training on similar cases is likely to improve detection rates.

Soft DSC loss was selected for the canine imaging model, which produced clinically 
acceptable vacuum bag contouring with a DSC of 0.952. Results indicated that the maximum measured MSD  0.82 mm, corresponded to an equivalent in-water depth of 0.08 mm; and hence, a negligible maximum shift in dose distribution ($<0.4\%$) at 10cm depth. As this model has the potential to save 30 minutes of planning time per patient, further work will involve clinical acceptance testing and implementation - under the condition that a human operator will still investigate all contouring prior to treatment. It is expected that little to no correction will be required during routine clinical use.


%\todo[inline]{Instead of saying (likely) here, a statement that leans on your PDD calculations done above would end up saying something like, the deviations in dose induced by using the auto-contoured vac bag model are never more than 0.1\% for any given ray line through the vac bag. Given many gantry angles are used, and there is not a significant systematic error within the vac-bag contouring, and under no circumstance was a vac-bag not contoured when in fact it was there, it is able to be rolled out for clinical use at SASH under the condition that a human operator still investigates the contours. It is expected that little to no correction should be needed during regular use}