\chapter{Results and Discussion}
\label{ch:results}

\section{Model 1: Pelvic imaging}
A total of 5 loss functions were assessed for ability to train a 2D U-Net with a
small dataset. As seen in Table \ref{table:loss_prostate}, dice similarity
coefficient (DSC), precision and sensitivity metrics were recorded on an
independent test dataset to measure model generalisability for each loss.

Binary cross entropy (BCE) achieved the strongest scores with a DSC of $>0.99$,
precision of $>0.99$, and sensitivity of $>0.99$. Patient contours had strong
agreement under BCE with an average organ specific DSC of $>
0.99(0.01)$;\footnote{Notation: $\bar{x}(\sigma)$ corresponds to a measurement
with mean value $\bar{x}$ and standard deviation $\sigma$.} however, BCE failed
to produce any positive predictions for the rectum, and only relatively large
bladder examples were identified - likely due to the BCE assumption that classes
are balanced \cite{Ronneberger_2015}, resulting in a local minima trapping
\cite{Khan2019}. However, relatively large bladder examples in the test dataset
received DSC scores higher than identical contours under w. soft DSC. (see
Figures \ref{fig:prostate_bladder} and \ref{fig:prostate_BCE_bladder}).

The standard soft DSC loss function also assumes equally weighted segmentation
classes throughout the data \cite{Sudre_2017}; and hence, failed to produce
positive predictions for the bladder and rectum. However, patient contouring had
a mean DSC $0.998(0.001)$. Research has indicated that the DSC is equivalent to
the harmonic mean of recall (sensitivity) and precision \cite{Bebis2019}; and
hence, weighs both equally \cite{Bebis2019} - contributing to the class
imbalance problem as the majority of output pixels are negatives. It is not
possible to control the trade-off between false-positives and false-negatives
directly within the standard soft DSC loss formulation \cite{taghanaki2018}; as
expected, we observe a bias that favours negatives due to their
over-representation in model output \cite{taghanaki2018}. To control for class
imbalance in pelvic CT imaging (where boundaries between OARs can be poorly
defined \cite{Liu_2020}) a loss function that enforces a higher penalty for
false-negative values may reduce under-segmentation and improve identification
in relatively smaller OAR examples \cite{taghanaki2018}. An additional point of
warning to note is that the soft DSC does not include true-negatives in its
calculation, hence specificity was not optimised directly \cite{taghanaki2018}.

Conversely, the weighted soft DSC was the only loss attempted that was able to
optimise for all organs in the contour space (see Figures
\ref{fig:prostate_patient}, \ref{fig:prostate_bladder}, and
\ref{fig:prostate_rectum}), and was selected as the final model. A simplified
combination BCE and weighted soft DSC loss was also attempted (see
\cite{taghanaki2018}) after experiments revealed BCE performance was superior to
the standard soft DSC metric, and contoured relatively large bladder examples
more accurately (i.e. higher DSC and sDSC) than the weighted soft DSC. However,
only patient contours were produced under the combination loss - reinforcing that
scalar selection to optimally balance a linear combination of loss functions is
a non-trivial task, dependent on the data distribution \cite{Bertels2019}.
 
Focal Tversky loss (see \cite{Khan2019}) exceeded both BCE and the combination
loss in sensitivity; however, performed poorly on average volumetric overlap
(DSC). Tversky loss successfully identified rectum and bladder contours in
almost all cases; however, segmentation masks contained many false-positive
results, with additional groupings that neither resembled nor were spatially
close to the OAR in question. Although Tversky loss aims to improve the
trade-off between sensitivity and precision compared to DSC for highly
imbalanced data (i.e. by weighting to penalise false-negatives higher than
false-positives) \cite{taghanaki2018}; results in Table
\ref{table:loss_prostate} indicate sensitivity was lower when compared to BCE
and the weighted soft DSC loss. However, Focal Tversky was the only loss
function to perform higher in sensitivity than precision - consistent with a
higher weighting on false-negatives as described in the literature
\cite{Khan2019}.

\begin{table}[h]
\footnotesize
\caption{Loss evaluation on independent test dataset for pelvic imaging}
% title of Table
\centering
% used for centering table
\begin{tabular}{c c c c}
% centered columns (4 columns)
\hline\hline
%inserts double horizontal lines
Loss & DSC & Precision & Sensitivity \\ [0.5ex]
% inserts table
%heading
\hline
% inserts single horizontal line
BinaryCrossentropy (BCE) & 0.995 & 0.999 & 0.991 \\
soft DSC & 0.986 & 0.999 & 0.972 \\
\textbf{w. soft DSC} & \textbf{0.994} & \textbf{0.997} & \textbf{0.991} \\
BCE + 2(w. soft DSC) & 0.985 & 0.999 & 0.971 \\
FocalTversky & 0.962 & 0.941 & 0.987 \\
% [1ex] adds vertical space
\hline\hline
%inserts single line
\end{tabular}
\label{table:loss_prostate}
% is used to refer this table in the text
\end{table}


The final model for pelvic imaging was selected at the 140th epoch, under
weighted soft DSC. As seen in Figure \ref{fig:prostate_metrics}, validation loss
plateaued at 130 epochs, with DSC value reaching a maximum at 140 epochs.
Although a smoothed representation of loss for both validation and training data
would be monotonically decreasing, this is not the case for DSC, precision, and
sensitivity metrics. Section \ref{ch:method-architecture} discussed the use of
BCE weight initialisation; hence, after switching from BCE to weighted soft DSC,
there is a significant change in loss topology. Examination of early model
predictions after BCE initialisation indicated a change in state from placing
minimal significance on bladder and rectum cases, to gradient updates dominated
by changes to these organs under weighted soft DSC. It is likely that feature
representations were subsequently perturbed significantly to minimise the new
loss - temporarily decreasing DSC and precision. A similar pattern emerged for
DSC value under Focal Tversky; however, the initial DSC value is comparatively
lower than under weighted soft DSC - indicating that convergence after three
epochs of BCE is sensitive to randomly sampled weight initialisation
\cite{Ronneberger_2015}. Total inference time per patient was approximately 3
seconds.

\begin{figure}[H]
	\begin{center}
		\hspace*{-1.2cm}\includegraphics[width=1.15\textwidth]{figures/prostate_metrics_all}
		\caption{\textbf{A)} Model training metrics for pelvic imaging via weighted
      soft DSC loss (w. soft DSC). Final model selected at epoch 140 due to
      validation loss plateau. Metrics begin post binary cross entropy (BCE)
      weight initialisation (3 epochs). Training time of 9~hours. \\
		\textbf{B)} Soft dice similarity coefficient (soft DSC) loss \\
		\textbf{C)} Combination binary cross entropy (BCE) and weighted soft dice
    similarity coefficient (w. soft DSC) loss\\
		\textbf{D)} Focal Tversky loss}
		\label{fig:prostate_metrics}
	\end{center}
\end{figure}


\begin{figure}[H]
	\begin{center}
		\includegraphics[width=1.0\textwidth]{figures/prostate_patient}
		\caption{Representative output for patient. Model 1 - trained via weighted
      soft dice (w. soft DSC) loss - 140 epochs.  Truth contour (yellow),
      prediction contour (red). Mean surface distance (MSD) mm.}
		\label{fig:prostate_patient}
	\end{center}
\end{figure}



\begin{figure}[H]
	\begin{center}
		\includegraphics[width=1.0\textwidth]{figures/prostate_bladder}
		\caption{Representative output for bladder: Model 1 - trained via weighted
      soft dice (w. soft DSC) loss - 140 epochs. Truth contour (yellow),
      prediction contour (red). Mean surface distance (MSD) in mm. sDSC
      \cite{Nikolov_2018} calculated at an organ specific tolerance of $\tau$ =
      1.46 mm, the 95th percentile mean surface distance between expert
      observers \cite{Roach_2019}.}
		\label{fig:prostate_bladder}
	\end{center}
\end{figure}

In Figure \ref{fig:prostate_bladder}, weighted soft DSC underestimated the
posterior aspect of larger bladder examples - a recurrent limitation of the
model. Figure \ref{fig:prostate_bladder} C) shows the model was correctly able
to identify when contours were not present, indicating the strong negative
predictive validity of the model - quantified by the high sensitivity score
recorded on Table \ref{table:organ}.

\begin{figure}[H]
	\begin{center}
		\includegraphics[width=1.0\textwidth]{figures/prostate_BCE_bladder}
		\caption{Representative model output for bladder: Trained via binary cross
      entropy loss - 78 epochs. Truth contour (yellow), prediction contour
      (red). Mean surface distance (MSD) in mm. sDSC \cite{Nikolov_2018}
      calculated at an organ specific tolerance of $\tau$ = 1.46 mm, the 95th
      percentile mean surface distance between expert observers
      \cite{Roach_2019}.}
		\label{fig:prostate_BCE_bladder}
	\end{center}
\end{figure}

Model 1 under w. soft DSC was unable to identify rectum contours containing
hollow regions (18\% of rectum contours) as seen in Figure
\ref{fig:prostate_rectum} D. It is suspected that further training on a
distribution of similar cases may improve performance. All other OAR examples in
the test dataset were correctly identified. As seen in Figure
\ref{fig:prostate_rectum} and Table \ref{table:organ}, DSC values for rectum
contours were lower on average when compared to the bladder - consistent with
expert IOV \cite{Roach_2019} and other models in the literature \cite{Liu_2020,
Kazemifar_2018, Wong2020}.



\begin{figure}[H]
	\begin{center}
		\includegraphics[width=1.0\textwidth]{figures/prostate_rectum}
		\caption{Representative output for rectum: Model 1 - trained via weighted
      soft dice (w. soft DSC) loss - 140 epochs. Truth contour (yellow),
      prediction contour (red). Mean surface distance (MSD) in mm. sDSC
      \cite{Nikolov_2018} calculated at an organ specific tolerance of $\tau$ =
      6.99 mm, the 95th percentile mean surface distance between expert
      observers \cite{Roach_2019}.}
		\label{fig:prostate_rectum}
	\end{center}
\end{figure}

\section{Model 2: Canine imaging}

Three loss functions were attempted for vacuum bag segmentation in canine
imaging, as seen in Table \ref{table:loss_vet}. Soft DSC outperformed both BCE
and focal Tversky on DSC and precision values. Focal Tversky had the highest
sensitivity (0.97) as expected \cite{Khan2019}, with BCE second (0.95). The
final model was selected at 100 epochs under soft DSC loss. Representative model
output is presented in Figure \ref{fig:vet_vacbag}. Soft DSC loss showed
clinically acceptable agreement with ground truth vacuum bag contours - and was
able to handle both negative (E) and small contour (B) examples in the test
dataset (see Figure \ref{fig:vet_vacbag}).

\begin{table}[h]
\footnotesize
\caption{Loss evaluation on independent test dataset for canine imaging}
% title of Table
\centering
% used for centering table
\begin{tabular}{c c c c}
% centered columns (4 columns)
\hline\hline
%inserts double horizontal lines
Loss & DSC & Precision & Sensitivity \\ [0.5ex]
% inserts table
%heading
\hline
% inserts single horizontal line
BinaryCrossentropy & 0.901 & 0.935 & 0.954 \\
\textbf{soft DSC} & \textbf{0.952} & \textbf{0.953} & \textbf{0.953} \\
FocalTversky & 0.930 & 0.906 & 0.969 \\
% [1ex] adds vertical space
\hline\hline
%inserts single line
\end{tabular}
\label{table:loss_vet}
% is used to refer this table in the text
\end{table}


\begin{figure}[H]
	\begin{center}
		\hspace*{-1.2cm}\includegraphics[width=1.15\textwidth]{figures/vacbag_metrics_combined}
		\caption{\textbf{A)} Model training metrics for canine imaging via soft dice
      similarity coefficient (soft DSC) loss. Final model selected at epoch 100
      due to validation loss plateau. Training time 6 hours.\\
		\textbf{B)} Binary cross entropy (BCE) loss\\
		\textbf{C)} Focal Tversky loss}
		\label{fig:vet_metrics}
	\end{center}
\end{figure}

Typically, it is expected that training metrics will overstate a model's
predictive capacity, as parameter values are updated to fit the training data
distribution. However, Figure \ref{fig:vet_metrics} indicated that at many
stages, validation loss was lower than training loss. The literature states two
possible contributing factors: 1) Dropout layers regularise only on the training
data, and hence the full architecture is only available for inference on the
validation and testing sets \cite{srivastava2014}. Additionally, batch
normalisation parameters are tuned to normalise activations on the test dataset
- and are fixed during validation and testing \cite{santurkar2018}. 2) Due to
the small dataset used in this study, variation in the validation set may itself
be small compared to the training data; hence, if the validation distribution is
centred about the mean of the training data, the validation dataset would be
relatively `easier' to infer \cite{Bishop}. Although Tversky loss continued to
decrease for both the training and validation sets over the epochs tested, a DSC
validation plateau (see oscillation in Figure \ref{fig:vet_metrics} C))
triggered early stopping.

\begin{figure}[H]
	\begin{center}
		\includegraphics[width=1.0\textwidth]{figures/vet_vacbag}
		\caption{Representative output for vacuum bag: Model 2 - trained via soft
      DSC loss. Truth contour (yellow), prediction contour (red). Mean surface
      distance (MSD) mm.}
		\label{fig:vet_vacbag}
	\end{center}
\end{figure}

\section{Clinical relevance}
The pelvic imaging model showed a strong agreement with observers for patient
contours, with an average DSC value of $0.998(0.001)$, and a mean surface
distance (MSD) of $0.002(0.005)$ mm, as seen in Table \ref{table:organ}. Bladder
contours had a DSC score of $0.9(0.2)$ and MSD of $1(3)$ mm. Volumetric rectum
agreement was considerably lower, with a DSC of $0.7(0.1)$, and a MSD of $1.(2)$ 
mm.

In comparison, organ-specific tolerances used for these OARs (95th percentile
MSD between experts - i.e. top 95\% expert variance) were calculated to be 1.5
mm and 7.0 mm for the bladder and rectum, respectively \cite{Roach_2019,
Nikolov_2018}. Large variances in bladder MSD corresponded to predictions that
under-segmented the posterior aspect of the bladder (as seen in Figure
\ref{fig:prostate_bladder} A-B). However, the vast majority of bladder surfaces
were contoured correctly within expert IOV - with a mean sDSC $0.9(0.2)$.
Rectum contours that were correctly identified also had a 
higher proportion of surface points that did not need correction to be within expert
IOV, with a sDSC of $0.9(0.1)$ recorded. Although, 18\% of rectum contours
required full-manual segmentation - reducing the models utility as QA tool.

Patient and rectum contours are clinically acceptable with DSC $>0.70$
\cite{Roach_2019}. However, experts have been reported to achieve similarities
much higher than this, with DSC 0.93 $\pm$ 0.03, MSD 0.99(0.30) mm for the
bladder and DSC 0.81 $\pm$ 0.07, MSD 2.862(2.066) mm for the rectum
\cite{Roach_2019}. Indicating that although bladder and rectum contours produced
may be clinically acceptable, performance falls short of expert IOV.

State-of-the-art U-Net implementations have recently been able to achieve DSC
values of 0.95 and 0.92 for the bladder and rectum, respectively
\cite{Kazemifar_2018}. However, 85 CT patient scans were included in this
dataset, compared to the 15 included in this study. It is suspected that
increasing the number of patients in this study would produce higher
generalisability in model performance (reported to scale logarithmically with
dataset size \cite{Nemoto_2020}) as well as provide a broader validation and
test distribution - improving the reliability and robustness of performance
metrics \cite{Ronneberger_2015}.

Model 2 produced vacuum bag contours with an average DSC of 0.952(0.001) and MSD
of 0.2(0.3) mm. Note that the vacuum bag has an electron density of
approximately 0.1\% of water \cite{Park}. The maximum MSD in the test dataset
was measured to be 0.82 mm, corresponding to an equivalent in-water depth of
0.08 mm. Hence, a negligible maximum shift in dose distribution ($<0.03\%$) is
expected at 10 cm depth under the contours produced by this model (see
\ref{fig:pdd}). If acceptance testing validates vacuum bag segmentations, model 2
has the potential to save approximately 30 minutes in treatment planning time
per patient.

\begin{table}[H]
\footnotesize
\caption{Organ specific evaluation for proposed models on independent test dataset}
\centering
\begin{tabular}{c c c c c c}
\hline\hline
Organ: Mean(Std) & sDSC ($\tau$) & DSC  & MSD (mm) & Sensitivity & Specificity \\ [0.5ex]
\hline
\textbf{Pelvic imaging} &              &              &              &       &       \\
Patient   &             & 0.998(0.001) & 0.002(0.005) & 0.997 & 0.999 \\
Bladder ($\tau$ 1.46 mm) [2] & 0.876(0.177) & 0.864(0.221) & 1.075(2.986) & 0.786 & 0.999 \\
Rectum ($\tau$ 6.99 mm) [2]  & 0.922(0.138) & 0.670(0.121) & 1.120(2.143) & 0.619 & 0.999 \\
Average                  &             & 0.994(0.153) & 0.409(1.604) & 0.991 & 0.999 \\ \\
\textbf{Canine imaging} &              &              &              &       &       \\
Vacbag    &              & 0.952(0.001) & 0.176(0.275) & 0.953 & 0.995\\
\hline\hline
\end{tabular}
\label{table:organ}
\end{table}


\section{Limitations and future work}

A common challenge in deep-learning applications to medical imaging is small
data set sizes \cite{Ronneberger_2015} as limited data reduces model
generalisability \cite{Shen2017}. With state-of-the-art implementations using up
to 1000 patients per study \cite{Nikolov_2018}, it is expected that bladder and
rectum segmentation could improve significantly with the broader distribution
provided by a larger dataset.

Additionally, organ-specific tolerances used in this study were calculated from 
Roach et al. from 15 expert observers (9 of which were radiation oncologists) by
averaging over a cohort of 5 patients \cite{Roach_2019}. Reliability in these
values could be improved by surveying a broader range of experts and patients. However,
parallel limitation in found in this study, where standard deviations
are calculated over small testing patient cohorts - and may not be
representative of the greater population.

Only one attempt to correlate sDSC with the time required for contour correction
was found in a search of all papers including the term `surface dice similarity
coefficient' on PubMed. Therefore, more work is required to assess the utility
of sDSC in a clinical workflow. Alternative surface-based metrics have also been
presented; for example, the estimated added path length in Vaassen et al. (seen
in Figure \ref{fig:vaassen}) \cite{Vaassen_2020}. However, Vaassen et al.
compared sDSC and estimated added path length under an x-y voxel-size tolerance,
rather than under organ specific IOV tolerances \cite{Vaassen_2020} - limiting
comparisons to this study.

Additionally, there is an opportunity to investigate the barriers and
limitations in designing a soft surrogate sDSC metric that can be optimised
directly during training. In its current implementation, the surface dice
similarity coefficient accept only binary input data; and hence, gradients are
not defined when used as a cost function.

Furthermore, our ultimate responsibility lies in improving patient outcomes.
Hence, there is an opportunity to correlate DSC and sDSC performance with
changes to dose distribution when compared with plans developed under expert
contouring. A potential advantage of sDSC compared to DSC is the stronger
correlation with the time required for contour correction - however, correlation
with dose shift is also likely to be an important clinical indicator.

Current studies are researching potential improvements in medical image
segmentation under 3D U-Net models. Although this study focused on a 2D
implementation due to the clinical barriers inherent in 3D models, more work is
still required to quantify the potential for performance improvement.
For instance, a 3D model may provide axial context to improve segmentation on
rectal regions containing gas - especially in the case that contextual slices
contain solid matter.
