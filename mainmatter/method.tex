\chapter{Method}
\label{ch:method}

\section{Datasets}
\label{ch:method-dataset}
This project focused on building an automated segmentation solution for radiotherapy. Two applications were targeted: A multi-organ segmentation model for pelvic imaging (with patient, bladder and rectum contouring); and a single structure model for vacuum bag contouring in canine imaging. Anonymised pelvic imaging data was provided by Riverina Cancer Care Centre from active prostate cancer RT patients over multiple stages of treatment. Canine imaging data was provided by the Small Animal Specialists Hospital (SASH) and contained variable cancer locations and patient orientations. All input data consisted of raw diagnostic CT images acquired with a 512 x 512 matrix. Pelvic imaging scans were comprised of 1.37 mm x 1.37 mm x 2 mm voxels; while canine imaging scans contained variable spacings across patients, with an average voxel size of 0.85 mm x 0.85 mm x 1.91 mm.

Patient scans were exported from the Monaco treatment planning system
to DICOM format, from which image-structure pairs were extracted, transformed from patient-space to a non-dimensional matrix-space, and saved individually as model input-output arrays for further processing. Initial modelling was attempted with contours extracted on-the-fly from DICOM files; however, this resulted in a significant CPU bottleneck which limited GPU capacity during training.

In addition, any advantage of removing the intermediate file processing step (DICOM to array) was made redundant upon determining that significant data cleaning would be required for contour consistency.
For instance, vacuum bag segmentation was often incomplete in patient scans, as only clinically relevant locations included full contouring; this may satisfy clinical requirements, however, consistent labels
%\todo{\ldots consistent labels on the slices used as inputs to the model\ldots}
are required for machine learning. The final data pipeline was designed to handle filenames (pointing to arrays) as the primary method of matching an input with the ground truth. Each filename was read into memory if it belonged to the current batch; removing memory constraints on dataset size, as only a single batch populated the RAM at each training step.

A total of 15 patients were used for pelvic imaging, corresponding to 1991 total input instances. Data was split at the patient level to enforce independence across training, validation and test datasets. 12 patient scans were used for training, while validation and testing used 2 and 1, respectively. The complete data distribution is provided in Table \ref{table:data_prostate}. Patient contours were present in all input data, while the bladder was present in 28\% of slices, and rectum in 37\%. A significant pixel-wise class imbalance existed between structures in pelvic imaging, as the multi-organ segmentation output space was 512 x 512 x 3. Patient pixels corresponded to a total of 5.2\% of all output pixel in the data, with 0.08\% and 0.02\% corresponding to bladder and rectum.

\begin{table}[h]
\footnotesize
\caption{Data distribution for pelvic imaging.}
\centering
\begin{tabular}{c c c c c}
\hline\hline
& Training & Validation & Testing & Total  \\ [0.5ex]
\hline

Images(Patients) & 1751(12) & 282(2) & 138(1) & 1991(15) \\
 \\
 \hline\hline
		 & Images total (\%) & Pixel-image ratio (\%)& Pixel-output ratio (\%) & Pixels total (\%)\\ [0.5ex]
\hline
Patient  & 100 & 15.6  &  5.21 & 5.21\\
Bladder  & 28.0 & 0.862 & 0.287 & 0.081\\
Rectum   & 37.0 & 0.172 & 0.057 & 0.021\\
\hline\hline
\end{tabular}
\label{table:data_prostate}
\end{table}


For vacuum bag segmentation, a total of 26 patients were used, with 21, 3, and 2 corresponding to training, validation and testing, respectively. Vacuum bag structures were present in 70\% of total input images, as seen in Table \ref{table:data_vet}.

\begin{table}[h]
\footnotesize
\caption{Data distribution for canine imaging.}
\centering
\begin{tabular}{c c c c c}
\hline\hline
& Training & Validation & Testing & Total  \\ [0.5ex]
\hline

Images(Patients) & 1912(21) & 340(3) & 187(2) & 2439(26) \\
 \\
 \hline\hline
		 & Images total (\%) & Pixel-image ratio (\%)& Pixel-output ratio (\%) & Pixels total (\%)\\ [0.5ex]
\hline
Vacbag   & 70.0& 13.4  & 13.4  &  9.4 \\

\hline\hline
\end{tabular}
\label{table:data_vet}
\end{table}



Significant data augmentation was used to increase the effective size of our dataset and to combat over-fitting via regularisation.
Augmentation was performed on-demand for each input-output pair in a batch, and sampled from a random uniform distribution with probability listed below for each type. 50\% of total training data was selected for augmentation per epoch. Augmentations included: Left-right image inversion ($P_{val}=0.5$), random image cropping and resizing ($P_{val}=0.33$, minimum crop size 500 x 500), elastic deformations ($P_{val}=0.33$, with ($\alpha$, $\sigma$) pairs selected from (1201, 10), (1501, 12), and (991, 8)), affine transformations ($P_{val}=0.33$, $\alpha_{max}=20$), and Gaussian noise ($P_{val}=0.33$, $\mu=0$, $\sigma_{max}=0.3$). Furthermore, all input data (including test data) was normalised with respect to the training and validation dataset distributions prior to augmentation. Randomly sampled transformations are included in Figure \ref{fig:augment}.

\begin{figure}[H]
	\begin{center}
		\includegraphics[width=1\textwidth]{figures/augment}
		\caption{Training data augmentation for single input image with random sampling of parameters: image crop and resize, affine transformation, elastic deformation, and combined transformations. Each matching contour set is augmented under an identical transformation. An individual transformation type has $P_{val}=0.33$ of occurring. Additional augmentations not shown: Left/right inversion and Gaussian noise.}
		\label{fig:augment}
	\end{center}
\end{figure}

\section{Model architecture}
\label{ch:method-architecture}
A 2D U-Net architecture was designed  with 7 levels, consisting of 6 encoding and 6 decoding blocks, outlined in Figure \ref{fig:model}. The model accepts a full resolution (512 x 512) CT image as input, and outputs selected contours in the original resolution by the use of padded convolutions, each of which is followed by batch normalisation and ReLU activation. Each encoding block performs a repeated sequence of 3 x 3 convolution (increasing feature channels). Extracted feature maps are passed via the skip connection in one pathway, while a 3 x 3 convolution with a stride size of 2 results in the halving of resolution, before features are passed to the next encoding block.

Conversely, each decoding block upsamples input via a 3 x 3 2D transposed convolution with stride size of 2. Upsampled feature maps are then concatenated with skip connections. Dropout is selectively performed with a probability value of 20\%, before additional 3 x 3 convolution sequences reduce the feature channels. Finally, multi-organ segmentation can be controlled via the C output variable, corresponding to the number of segmentations (or channels) specified in the final 1 x 1 convolutional layer. A sigmoid activation function is used to account for the non-mutually exclusive nature
of pixel-wise binary classification on anatomic structures (i.e. a voxel can belong to multiple structures).

\begin{figure}[H]
	\begin{center}
		\hspace*{-1.3cm}\includegraphics[width=1.15\textwidth]{figures/model_diagram}
		\caption{Modified 2D U-Net architecture: Composed of encoding (blue) and decoding blocks (yellow). MaxPooling layers replaced by strided convolution. Added batch normalisation and final sigmoid activation. Tensor dimensions (Batch size, X, Y, Channels) are included for each connection. Internal layers of encoding blocks (blue) and decoding blocks (yellow) are included under the high-level overview.}
		\label{fig:model}
	\end{center}
\end{figure}

The model was trained using the Adam (Adaptive momentum estimation) optimisation algorithm \cite{kingma2014} with an initial learning rate of $10^{-5}$, a batch size of 1 for pelvic imaging, and 3 for canine imaging. Model training was scheduled to conclude when validation loss had not improved for a period of 20 epochs. In addition, learning rate decay was triggered by a validation loss plateau period of 3 epochs. Initial model weights were determined via `He' kernel initialisation), which samples from a zero mean Gaussian distribution with variance $\sigma=\sqrt{2/N}$ (as in Ronneberger et al \cite{Ronneberger_2015}), where N is the incoming nodes for a single activation (i.e. for n x n convolution over M feature maps, N = n x n x M). In addition, training for pelvic imaging was accelerated by adopting the strategy of Bertels et al., to further initialising model parameters via 3 epochs of training with binary cross entropy.

A total of 5 loss functions were assessed for pelvic imaging: Binary Cross entropy (BCE), soft dice similarity coefficient (soft DSC) \cite{Bertels2019}, weighted soft dice similarity coefficient (w. soft DSC or generalised dice loss) \cite{Sudre_2017}, a combination loss BCE + 2 (w. soft DSC) \cite{taghanaki2018}, and focal Tversky loss \cite{Zhu_2018, Khan2019, abraham2018}. In contrast, weighted soft DSC was not attempted for canine imaging due to the single segmentation output. Weights for the w. soft DSC were inversely proportional to the number of positive pixels in each ground truth segmentation (as described by Sudre et al. \cite{Sudre_2017}) - forcing the model to place a higher priority on infrequent or relatively smaller OARs \cite{Sudre_2017}. Focal Tversky  loss (as described in Khan et al. \cite{Khan2019}) was calculated with $\gamma = \frac{4}{3}$, $\alpha=0.7$, and $\beta=0.3$, as this was reported to produce the highest performance in the literature \cite{Khan2019} - penalising false-negative outputs ($\alpha > \beta$) and focusing on classes with the lowest accuracy ($\gamma > 1$) \cite{Khan2019}. A TensorFlow implementation of each loss function is included in the PyMedPhys library.

\section{Clinical performance}

Final model performance was evaluated on an independent test dataset via both DSC and sDSC metrics. Organ specific tolerances $\tau$ for rectum and bladder contours were taken as the 95th percentile absolute mean surface distance in mm between expert observers from Roach et al., with 1.5 mm  measured for the bladder, and 7.0 mm for the rectum \cite{Roach_2019}. 

As seen in Figure \ref{fig:pdd} a 6 MV photon beam with 10 x 10 field has a maximum dose deposition at approximately 1.4cm in water \cite{Nurdin}. Beyond this point, dose drop off can be approximated in the linear regime with a gradient of 4\% per cm or 0.4\% per mm \cite{Nurdin}. For example, a bladder border tolerance of 1.5 mm corresponds to a maximum dose differential of 0.6\% at the boundary. While a rectum tolerance of 7.0 mm corresponds to a dose differential of 2.8\%. Additionally, the literature reports that clinically acceptable agreement between expert observers is a DSC $\geq$ 0.7 for the bladder and rectum \cite{Roach_2019}. These measurements and the percentage depth dose curve presented in Figure \ref{fig:pdd} were used to assess clinical acceptability of the contours produced by each model.


\begin{figure}[H]
	\begin{center}
		\includegraphics[width=0.8\textwidth]{figures/pdd}
		\caption{Percentage depth dose curve (water) for a 6 MV photon beam with 10 x 10 cm field setup. Figure redrawn from Nurdin et al. \cite{Nurdin}.}
		\label{fig:pdd}
	\end{center}
\end{figure}
