\chapter{Introduction} \label{ch:introduction}

\section{Background}
Approximately 18 million patients are diagnosed with cancer each year \cite{Bray2019}, with reports indicating that up to 50\% of all cases could benefit from radiotherapy (RT) for curative or palliative management of disease (optimal radiotherapy utilisation rate) \cite{Barton2014}. 
 Treatment planning for RT involves the optimisation of beam quality, field arrangements, dose distributions, and fractionation schedules. Optimisation aims to maximise tumour control probability and minimise normal tissue complication probability - i.e. maximising the therapeutic ratio of treatment \cite{iaea2016}. Contouring (or delineation) is a critical aspect of treatment planning, it describes the process of defining and classifying anatomical regions-of-interest (ROIs) within a patient from medical imaging data \cite{iaea2016}. Regions include: target volumes for treatment, associated error margins; as well as normal tissue regions (organs at risk - OARs) for which exposure needs to be minimised to avoid adverse side-effects of treatment \cite{iaea2016}. Once ROIs are delineated, they are used within the treatment planning system for the calculation of an optimum dose distribution (as seen in Figure \ref{fig:contour}). Therefore, contours are part of the primary geometry used to optimise treatment outcomes; and as such, accurate contouring is fundamental to the efficacy of RT \cite{Nikolov_2018}. \todo{Still want to work on wording here!}


\begin{figure}[!htb]
	\begin{center}
		\includegraphics[width=0.5\textwidth]{contour}
		\caption{Single posterior field setup for carbon ion radiotherapy treatment of pancreatic cancer. Multiple contours are outlined on diagnostic CT imaging for treatment planning. Colour map shows the dose distribution over patient anatomy. Figure redrawn from Dreher et al. \cite{Dreher2017}.}
		\label{fig:contour}
	\end{center}
\end{figure}

However, there are current limitations in clinical practice. Large intra- and inter-practitioner (observer) variability (IOV) exists in the definition of ROIs on medical imaging. IOV is a long-standing challenge in RT, and is frequently reported as the largest source of error in accurate treatment delivery \cite{Vinod_2016, tg100}.
Additionally, manual contouring is both time consuming and requires skilled experts \cite{Nikolov_2018}. For instance, research has estimated that a radiation oncologist (RO) needs between 90 -- 120 minutes to delineate pelvic OARs in a cervical cancer patient \cite{Liu_2020}. The process is often computer aided and automatic OAR segmentation tools (i.e. deformable image registration or Atlas-based methods) have both reduced contouring times and improved consistency between expert observers \cite{Vinod_2016}. However, atlas-based methods still require significant manual editing \cite{Nikolov_2018}, and experience difficulty with small organ volumes, regions with poor contrast for differentiation, or high variability in size or location - such as pelvic OARs in prostate and cervical cancer \cite{Schreier_2020, Liu_2020}. Critically, current automatic solutions still present a barrier to the adoption of future technologies that would require fast contouring \cite{Nikolov_2018}. For instance, adaptive radiotherapy has shown the potential to deliver a new standard of care for RT patients by updating treatment plans to daily changes in patient anatomy \cite{Nikolov_2018}. 

In contrast, deep-learning (DL) algorithms have shown significant performance improvements over atlas-methods both in terms of accuracy and time for contour generation \cite{Liu_2020}. However, the implementation of DL models in clinical environments remains challenging; particularly due to limitations in quantitatively assessing model performance in comparison to expert performance and IOV \cite{Nikolov_2018}. Typical metrics used to quantify the similarity between model and expert contours are volumetric in nature, hence volume overlap tends to be the focus of evaluation \cite{Nikolov_2018}. Recent studies have introduced surface-based performance metrics that aim to provide direct information on the fraction of surface points that require correction to be within IOV tolerances (specific to each OAR) \cite{Nikolov_2018, Vaassen_2020}, and may provide a stronger correlation with time required for contour correction \cite{Vaassen_2020}.

U-Net is a type of deep-learning architecture that leverages multi-resolution analysis to perform state-of-the-art segmentation in medical imaging research \cite{Kazemifar_2018, Zhu_2018} This study presents two independent U-Net models.

\section{Aim}


\textbf{Model 1 - Pelvic imaging QA tool}

Model 1 was designed to fulfil the need for contouring to become part of regular quality assurance, and to evaluate the ability of U-Net to achieve expert level performance - as defined by Nikolov et al.'s surface dice similarity coefficient (sDSC). We focused on pelvic CT imaging scans, automatically contouring the patient, bladder and rectum volumes. These organs were selected for their relative ease of contouring, compared to other structures (i.e. head and neck), with the intent to act as a template for future work extending to other organs

% The model then compares predicted contours with those created by an expert clinician; and provides feedback on the volumetric overlap (dice similarity coefficient - DSC), as well as the percentage of surface points that deviate by a distance larger than the IOV associated with each structure (sDSC). We present both metrics for comparison, with models trained over a variety of loss functions common to medical image segmentation. METHOD

\textbf{Model 2 - Automatic segmentation of vacuum bag structures in canine imaging}

Model 2 was designed to automatically contour vacuum bag structure in canine imaging. Vacuum bag structures are reported to take approximately 30 minutes per patient to contour; hence, this model aims to automate a time-consuming aspect of RT treatment that is processed manually at SASH veterinary clinic.


